\documentclass[dvipdfmx, fleqn, titlepage]{jsarticle}

%% packages and libraries
\usepackage{amsmath, amssymb, amsfonts, mathtools, cases, physics, mathrsfs}
\usepackage{physics}
\usepackage{fancyhdr, lastpage}
\usepackage{titlesec}
\usepackage{hyperref}
\usepackage{url}
\usepackage{pxjahyper}
\usepackage{overcite}
\usepackage{pxrubrica}
\usepackage[table]{xcolor}
\usepackage{longtable, float, multirow, array, listliketab, enumitem, tabularx}
\usepackage{tcolorbox}
\tcbuselibrary{skins, listings, breakable}
\usepackage{listings, plistings}
\usepackage{comment}
\usepackage{graphicx}
\usepackage{subcaption, wrapfig}
\usepackage{tikz}
\usetikzlibrary{calc, patterns, decorations, decorations.pathmorphing, angles, backgrounds, shadows}
\usepackage{pdfpages}
\usepackage{import, grffile}
\usepackage{standalone}
\usepackage{xparse}
\usepackage[T1]{fontenc}
\usepackage{textcomp}
\usepackage[utf8]{inputenc}
\usepackage{lmodern}
\usepackage{mathptmx}
\usepackage[scaled]{helvet}
\renewcommand{\ttdefault}{pcr}
\usepackage[deluxe]{otf}
\usepackage[noalphabet]{pxchfon}
\usepackage{bm}
\usepackage{lscape}
\usepackage{siunitx}
\usepackage{bigstrut}


%% set up for hyperref
\hypersetup{
	bookmarksnumbered = true,
	hidelinks,
	colorlinks = true,
	linkcolor = black,
	urlcolor = cyan,
	citecolor = black,
	filecolor = magenta,
	setpagesize = false,
	pdftitle = {},
	pdfauthor = {R.Morita},
	pdfkeywords = {},
	}


%% set up for siunitx
\sisetup{%
	%detect-family = true,
	detect-inline-family = math,
	detect-weight = true,
	detect-inline-weight = math,
	%input-product = *,
	quotient-mode = fraction,
	fraction-function = \frac,
	inter-unit-product = \ensuremath{\hspace{-1.5pt}\cdot\hspace{-1.5pt}},
	per-mode = symbol,%
	product-units = single,%
	}


%% fonts
\setminchofont{ipam.ttf}
\setgothicfont{ipag.ttf}
\fontsize{14Q}{30H}


%% layout
%space between letters
\kanjiskip 0zw plus 1zw minus 0zw
\xkanjiskip 0.25zw plus 1pt minus 1pt
%line skip
\setlength{\lineskiplimit}{2pt}
\setlength{\lineskip}{2pt}
%indent
\setlength{\parindent}{1zw}
\setlength{\mathindent}{5zw}


%header & footer
\fancypagestyle{jsarticledefault}{%
	%reset
	\fancyhf{}
	%layout
	\setlength{\textwidth}{453pt}
	\setlength{\hoffset}{0pt}
	\setlength{\oddsidemargin}{0pt}
	\setlength{\marginparsep}{18pt}
	\setlength{\marginparwidth}{18pt}
	\setlength{\textheight}{634pt}
	\setlength{\voffset}{0pt}
	\setlength{\topmargin}{4pt}
	\setlength{\headheight}{20pt}
	\setlength{\headsep}{18pt}
	\setlength{\footskip}{28pt}
	%header
	\renewcommand{\headrule}{}
	%footer
	\cfoot{\thepage}
	}


%put citation mark on index zone
\renewcommand{\citeform}[1]{[#1]}


%hyphen for math mode
\DeclareMathSymbol{\mhyphen}{\mathalpha}{operators}{`-}


%unit of formulas 〔roman〕
\newcommand{\unitis}[1]{%
	\text{〔\si{#1}〕}
	}

\title{
	知能システム論 \\
	第1回 レポート
	}
\begin{document}
\maketitle

教師付き学習,教師無し学習,強化学習の独自の例を一つずつ挙げ,
その有用性,実現可能性,社会的影響などについて論ずる。


\section{教師付き学習}

一例として,文章を校正するモデルを挙げる。
これは例えば,文章を入力とし,文法的に誤っている部分や事実に反する部分を指摘する
(文字ベースや単語ベースでそのインデックスを返す)
ようなモデルである。
有用性は言わずもがなであり,
出版物やニュース記事のライター及びその校正を行う人の業務を大きく効率化できる。
また,論文やブログ記事などあまり他人に校正されないようなもののチェックにも用いることができる。
このモデルを用いることで,文章の良さのようなものを誤っている文字数などで数値化できるようになり,
検索において「質のいい文章」を優先して出す,といったことも可能になる。
実現可能性は,文法エラー修正についてはある程度可能である一方,
事実に反する部分を見つけるところについては課題があると考える。
文法エラー修正は,誤字脱字の指摘程度なら
教師データが十分揃っていれば学習可能でかつ汎化性能も高くなると思われる。
しかし,事実に反するものの指摘は,モデルに一般常識を教えなければいけないため,
汎化性能を高くするのがかなり難しいと考えられる。



\section{教師無し学習}

低次元化によって計算速度を高め,検索を高速化する。
特に画像検索においては,画像のみで行おうとする場合どうしても高次元になってしまうので,
例えばwebサービスで用いる場合レスポンス速度がネックになる。
機械学習モデルの構築の際にレスポンス速度まで考慮するのは難しいため,
画像をembedするモデルと低次元化するモデルが別れていることに一定の価値はあると思われる。
また,画像embedderの学習には一般に時間がかかるが,
低次元化するためのモデルの学習は比較的高速である点も有利である。
以上のような有用性により,
機械学習を用いた検索システムの導入ハードルが下がるので,
より機械学習の導入が活発となり,社会的にもインパクトはある。
また,実現可能性についても,
PCAなどで十分良い圧縮が得られるので,
担保されているといえる。



\section{強化学習}

スポーツの戦略探索に用いることができると考えられる。
テニスを例に取って考えてみると,
1回球を打つだけでも球速や弾道の高低,また左右の方向など様々な選択肢があり,
自分や相手のいる位置なども考慮する必要がある。
このような無数の選択肢がある状況に対して,
人間は経験から次の行動を選択するが,
これを強化学習モデルに学習させることで
囲碁や将棋の例のように新たな戦術が生まれる可能性が高いといえる。
実現可能性については,機械学習モデルについてもそうだが,
学習環境の整備の方にも課題があると思われる。
人間・ラケット・球を物理的に正しく動かし,かつその処理を高速に行うこと,
また,疲れなどのモデリングを行うことはかなり難しいのではないかと考えられる。



\end{document}
