\documentclass[class=jsarticle, crop=false, dvipdfmx, fleqn]{standalone}
%% title
\author{37-196360 \quad 森田涼介}


%% packages and libraries
\usepackage{amsmath, amssymb, amsfonts, mathtools, cases, physics, mathrsfs}
\usepackage{physics}
\usepackage{fancyhdr, lastpage}
\usepackage{titlesec}
\usepackage{hyperref}
\usepackage{url}
\usepackage{pxjahyper}
\usepackage{overcite}
\usepackage{pxrubrica}
\usepackage[table]{xcolor}
\usepackage{longtable, float, multirow, array, listliketab, enumitem, tabularx}
\usepackage{tcolorbox}
\tcbuselibrary{skins, listings, breakable}
\usepackage{listings, plistings}
\usepackage{comment}
\usepackage{graphicx}
\usepackage{subcaption, wrapfig}
\usepackage{tikz}
\usetikzlibrary{calc, patterns, decorations, decorations.pathmorphing, angles, backgrounds, shadows}
\usepackage{pdfpages}
\usepackage{import, grffile}
\usepackage{standalone}
\usepackage{xparse}
\usepackage[T1]{fontenc}
\usepackage{textcomp}
\usepackage[utf8]{inputenc}
\usepackage{lmodern}
\usepackage{mathptmx}
\usepackage[scaled]{helvet}
\renewcommand{\ttdefault}{pcr}
\usepackage[deluxe]{otf}
\usepackage[noalphabet]{pxchfon}
\usepackage{bm}
\usepackage{lscape}
\usepackage{siunitx}
\usepackage{bigstrut}


%% set up for hyperref
\hypersetup{
	bookmarksnumbered = true,
	hidelinks,
	colorlinks = true,
	linkcolor = black,
	urlcolor = cyan,
	citecolor = black,
	filecolor = magenta,
	setpagesize = false,
	pdftitle = {},
	pdfauthor = {R.Morita},
	pdfkeywords = {},
	}


%% set up for siunitx
\sisetup{%
	%detect-family = true,
	detect-inline-family = math,
	detect-weight = true,
	detect-inline-weight = math,
	%input-product = *,
	quotient-mode = fraction,
	fraction-function = \frac,
	inter-unit-product = \ensuremath{\hspace{-1.5pt}\cdot\hspace{-1.5pt}},
	per-mode = symbol,%
	product-units = single,%
	}


%% fonts
\setminchofont{ipam.ttf}
\setgothicfont{ipag.ttf}
\fontsize{14Q}{30H}


%% layout
%space between letters
\kanjiskip 0zw plus 1zw minus 0zw
\xkanjiskip 0.25zw plus 1pt minus 1pt
%line skip
\setlength{\lineskiplimit}{2pt}
\setlength{\lineskip}{2pt}
%indent
\setlength{\parindent}{1zw}
\setlength{\mathindent}{5zw}


%header & footer
\fancypagestyle{jsarticledefault}{%
	%reset
	\fancyhf{}
	%layout
	\setlength{\textwidth}{453pt}
	\setlength{\hoffset}{0pt}
	\setlength{\oddsidemargin}{0pt}
	\setlength{\marginparsep}{18pt}
	\setlength{\marginparwidth}{18pt}
	\setlength{\textheight}{634pt}
	\setlength{\voffset}{0pt}
	\setlength{\topmargin}{4pt}
	\setlength{\headheight}{20pt}
	\setlength{\headsep}{18pt}
	\setlength{\footskip}{28pt}
	%header
	\renewcommand{\headrule}{}
	%footer
	\cfoot{\thepage}
	}


%put citation mark on index zone
\renewcommand{\citeform}[1]{[#1]}


%hyphen for math mode
\DeclareMathSymbol{\mhyphen}{\mathalpha}{operators}{`-}


%unit of formulas 〔roman〕
\newcommand{\unitis}[1]{%
	\text{〔\si{#1}〕}
	}


%% setting for listings
\newtcbinputlisting[auto counter]{\reportlisting}[3][]{%
	listing file = {#3},
	listing options = {language=python, style=tcblatex, numbers=left, numberstyle=\tiny},
	listing only,
	breakable,
	toprule at break = 0mm,
	bottomrule at break = 0mm,
	left = 6mm,
	sharp corners,
	drop shadow,
	title = Listings \thetcbcounter : \texttt{#2},
	label = #1,
	}

\begin{document}

\section*{宿題1}

\(\bm{x} \in \mathbb{R}^d\)についての関数\(f\)を,
\begin{equation}
    f(\bm{x}) = \frac{1}{2} \bm{x}^\mathrm{T} \bm{A} \bm{x}
\end{equation}
とする。
ここで,\(\bm{A}\)は正定値対称行列である。
この\(f\)の厳密直線探索の最適なステップ幅は,
\(f\qty(\bm{x}_k - \varepsilon_k \nabla f(\bm{x}_k))\)
を最小にする\(\varepsilon_k\)を考えることにより,
\begin{equation}
    \varepsilon_k
        = \frac{||\nabla f(\bm{x}_k)||^2}{\nabla f(\bm{x}_k)^\mathrm{T} \bm{A} \nabla f(\bm{x}_k)}
        = \frac{\bm{x}_k^\mathrm{T} \bm{A}^2 \bm{x}_k}{\bm{x}_k^\mathrm{T} \bm{A}^3 \bm{x}_k}
\end{equation}
で与えられる(\(\nabla f(\bm{x}) = \bm{A} \bm{x} \))。

いま,\(f\)の真の減少量を,
\begin{equation}
    g(\varepsilon_k) = f(\bm{x}_k) - \varepsilon_k \nabla f(\bm{x}_k) - f(\bm{x}_k)
\end{equation}
とすると,
Armijo規準より,\(\alpha \in \qty(0,\ 1)\)について,
\begin{equation}
    g(\varepsilon_k) \le \alpha \varepsilon_k g^\prime (0)
\end{equation}
が成り立つ。
ここで,\(g\)を\(\varepsilon_k\)についてTaylor展開することで,
\begin{equation}
    g^\prime (0) = - ||\nabla f(\bm{x}_k)||^2
\end{equation}
を得る。ここで,
\begin{align}
    f(\bm{x}_k - \varepsilon_k \nabla f(\bm{x}_k))
        & = \frac{1}{2} (\bm{x}_k - \varepsilon_k \nabla f(\bm{x}_k))^\mathrm{T} \bm{A} (\bm{x}_k - \varepsilon_k \nabla f(\bm{x}_k)) \\
        & = \frac{1}{2} \bm{x}_k^\mathrm{T} \bm{A} \bm{x}_k - \varepsilon_k \nabla f(\bm{x}_k)^\mathrm{T} \bm{A} \bm{x}_k + \frac{1}{2} \varepsilon_k^2 \nabla f(\bm{x}_k)^\mathrm{T} \bm{A} \nabla f(\bm{x}_k)
\end{align}
から,結局Armijo規準は次のように整理される。
\begin{align}
    & g(\varepsilon_k) \le \alpha \varepsilon_k g^\prime (0) \\
    & \qty(\frac{1}{2} \bm{x}_k^\mathrm{T} \bm{A} \bm{x}_k - \varepsilon_k \nabla f(\bm{x}_k)^\mathrm{T} \bm{A} \bm{x}_k + \frac{1}{2} \varepsilon_k^2 \nabla f(\bm{x}_k)^\mathrm{T} \bm{A} \nabla f(\bm{x}_k)) - \frac{1}{2} \bm{x}_k^\mathrm{T} \bm{A} \bm{x}_k \le - \alpha \varepsilon_k ||\nabla f(\bm{x}_k)||^2 \\
    & \qty(\frac{1}{2} \nabla f(\bm{x}_k)^\mathrm{T} \bm{A} \nabla f(\bm{x}_k)) \varepsilon_k^2 \le (1 - \alpha) ||\nabla f(\bm{x}_k)||^2 \varepsilon_k
\end{align}
\(\varepsilon_k > 0,\ \bm{A} > \bm{O}\)から,
\begin{equation}
    \varepsilon_k \le 2(1 - \alpha) \frac{||\nabla f(\bm{x}_k)||^2}{\nabla f(\bm{x}_k)^\mathrm{T} \bm{A} \nabla f(\bm{x}_k)}
\end{equation}
よって,\(\alpha = 1/2\)のArmijo規準を満たす最大の\(\varepsilon_k\)は,
\begin{equation}
    \varepsilon_k
        = \frac{||\nabla f(\bm{x}_k)||^2}{\nabla f(\bm{x}_k)^\mathrm{T} \bm{A} \nabla f(\bm{x}_k)}
        = \frac{\bm{x}_k^\mathrm{T} \bm{A}^2 \bm{x}_k}{\bm{x}_k^\mathrm{T} \bm{A}^3 \bm{x}_k}
\end{equation}
となり,厳密直線探索の最適なステップ幅と一致する。


\end{document}
